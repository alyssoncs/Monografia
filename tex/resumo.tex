\setlength{\absparsep}{18pt} % ajusta o espaçamento dos parágrafos do resumo
\begin{resumo}
	Segundo a NBR6028:2003, o resumo deve ressaltar o objetivo, o método, os resultados e as conclusões do documento. A ordem e a extensão destes itens dependem do tipo de resumo (informativo ou indicativo) e do tratamento que cada item recebe no documento original. O resumo deve ser precedido da referência do documento, com exceção do resumo inserido no próprio documento. (\ldots) As palavras-chave devem figurar logo abaixo do resumo, antecedidas da expressão Palavras-chave:, separadas entre si por ponto e finalizadas também por ponto.

	\textbf{Palavras-chave}: \iot{}. \Middleware{}.
\end{resumo}

\begin{resumo}[Abstract]
	According to NBR6028: 2003, the abstract should highlight the purpose, method, results and conclusions of the document. The order and extent of these items depend on the type of summary (informative or indicative) and the treatment each item receives in the original document. The abstract should be preceded by the reference of the document, with the exception of the abstract inserted in the document itself. (\ldots) Keywords should appear just below the abstract, preceded by the expression Keywords: separated by each point and also terminated by point.

	\textbf{Keywords}: \iot{}. \Middleware{}.
\end{resumo}
