\chapter{Solução proposta}\label{chap:solucao}

O trabalho em questão tem como objetivo a implementação do mecanismo que propaga os eventos de descoberta, conexão e desconexão que ocorrem no \stwopa do \mhub até a camada de aplicação do \cddl, de forma que os desenvolvedores possam projetar aplicações que se adaptem a tais eventos.
Visto que o \cddl somente exportava à camada de aplicação, eventos de leitura de dados de contexto.

Significando que dentre todos os objetos do tipo \sensordata criados no \stwopa, apenas aqueles com atributo ``\texttt{action}'' assumindo valor ``\texttt{READ}'' geravam eventos no \cddl que podiam ser percebidos pela aplicação (vide \autoref{subsub:s2pa}).

Este fato implica em algumas limitações no desenvolvimento de aplicações.
Imaginando um cenário onde uma aplicação necessite de certos dados providos por um \smartobj--- alocando recursos computacionais para processa-los.
Em uma eventual desconexão com o \smartobj, o fluxo de dados do sensor cessaria de ser entregue à aplicação, contudo, nenhum fluxo ou notificação do evento de desconexão em si seria entregue.
Tal aplicação não poderia decidir se o interrompimento do fluxo de dados se deu: devido a uma mudança na latência do envio de dados por parte do sensor, ou por uma desconexão; não podendo então decidir sobre a necessidade da desalocação de recursos.

Outra classe de aplicações de \iomt que seriam prejudicadas, são aquelas que interagem com \smartobjs no ambiente por outros meios além de conexões.
É o caso de aplicações de localização \textit{indoor} baseadas em \beacons \bluetooth, onde os \beacons são dispostos no interior de ambientes físicos e permanecem realizando \broadcast de sua presença.
Os dispositívos móveis não realizam tentativas de conexão, apenas percebem a presença destes \smartobjs e utilizam a intensidade do sinal capturado no momento.

Vale ressaltar que a implementação também deve permitir que tais eventos possam estar disponíveis para outras aplicações que estejam interessadas, utilizando para isso o \mqtt.
Ou seja---caso configurado desta forma---uma aplicação \mhubcddl pode mudar seu comportamento baseado em eventos que foram disparados a partir de interações entre \smartobjs e \smartphones remotos.

\section{Metodologia} \label{sec:metodologia}

O desenvolvimento do componente de \software foi realizado utilizando a metodologia ágil \textit{Feature-driven development}~\cite{coad:luca:lefebvre:1999}, dividido nas seguintes etapas:

\begin{alineas}
	\item transporte dos eventos gerados no \stwopa para o \cddl;

	\item definição de uma estrutura de tópicos onde os eventos serão publicados pelo \cddl, e posteriormente subscritos pelas aplicações interessadas;

	\item criação de uma \api para o consumo dos eventos.
\end{alineas}

Dentre as características que se considerou importantes, decidiu-se realizar a avaliação de desempenho e acurácia da solução.
As avaliações são realizadas através de simulações de cenários de uso, estes estão descritos no \autoref{chap:avaliacao}.

Como métricas da avaliação de desempenho, mede-se o tempo entre a ocorrência do evento e a sua respectiva notificação na camada de aplicação. Já as métricas de acurácia consistem na verificação da quantidade de notificação de eventos em relação a quantidade de eventos gerados.

\section{Requisitos de \software}

\subsection*{Requisitos funcionais}

Os requisitos funcionais definidos para a solução são:

\begin{alineas}
	\item notificação de eventos de descoberta, conexão e desconexão de \smartobjs para a camada de aplicação do \software;

	\item separação dos fluxos de eventos, provendo uma \api que permita a aplicação registrar interesse em cada tipo de evento individualmente;

	\item permitir que os eventos sejam acessíveis tanto para a aplicação que os gera, quanto para outras aplicações que registrem interesse;

	\item fornecer uma \api assíncrona para o recebimento de cada notificação dos eventos desejados.
\end{alineas}

\section{Implementação}

Esta seção descreve os detalhes de implementação da solução, abordando todos os tópicos descritos na \autoref{sec:metodologia}.

\subsection{Propagação de eventos do \stwopa para o \cddl}

Como descrito na \autoref{subsub:s2pa}, o \mhub encapsula todos os eventos em objetos do tipo \sensordata, estes objetos devem então ser propagados para o \cddl.
O \stwopa comunica-se com o \cddl através do componente \qocevaluator, como pode ser observado na \autoref{fig:mhub-cdll-architecture}~\cite{gomes:2017}.

A comunicação entre os componentes é feita através da biblioteca \eventbus\footnote{\url{http://greenrobot.org/eventbus/}}.
O \eventbus é uma biblioteca de eventos de código livre escrita em Java para a plataforma \android utilizando o padrão \pubsub, fornecendo um mecanismo central de comunicação simplificando a interação entre componentes da aplicação.

O \stwopa foi modificado então para publicar cada \sensordata no \eventbus, enquanto o \qocevaluator registra interesse em receber objetos do tipo \sensordata publicados no \eventbus.
O que efetivamente transfere todos os eventos gerados pelo \stwopa ao \cddl.

