\chapter{Avaliação quantitativa}\label{chap:avaliacao}

O objetivo deste capítulo é descrever as avaliações utilizadas na solução proposta. A avaliação tem como objetivo determinar a performance e acurácia do mecanismo de descoberta, conexão e desconexão de \smartobjs{}.

A avaliação será realizada por meio de experimentos, onde a performance será avaliada através da contagem de tempo desde que o \mhub{} efetivamente descobre um objeto até o momento em que a aplicação é informada sobre este evento. A acurácia da solução consistirá na verificação de quantas operações de conexão, desconexão e descoberta geraram eventos correspondentes para a aplicação.

\section{Experimento 1}\label{chap:avaliacao-experimento1}

Este experimento tem como objetivo avaliar a performance e acurácia da solução. Para tal, foi desenvolvido um cenário de uso, onde será possível determinar tais aspectos.

Este cenário de uso consiste em uma casa onde cada cômodo é equipado com um \beacon{} \ble{}, calibrado de forma que o sinal emitido não possa ser detectado fora do cômodo e configurado para emitir 10 anúncios por segundo. Uma pessoa portando um \smartphone{} é instruída a conduzir as atividades do dia a dia nesta residência. Este \smartphone{} executa uma aplicação desenvolvida com o \middleware{} \mhubcddl{} que detecta os anúncios dos \beacons{}, determinando em qual região da casa o indivíduo se encontra.

Ao entrar em um cômodo, o \beacon{} será encontrado pelo \mhubcddl{}, gerando um evento de descoberta no \stwopa{} que deverá ser propagado para a aplicação.

\subsection{Métricas}

Para este experimeto, cada anúncio detectado corresponde a um evento de descoberta gerado pelo \stwopa{}, este evento deve então ser propagado até a aplicação.

A fim de determinar a performance, a aplicação permanece constantemente anotando os \timestamps{} de cada evento de descoberta emitido pelo \stwopa{}, e os \timestamps{} de notificações destes eventos na aplicação---ambos em milisegundos. O primeiro \timestamp{} é subtraído do segundo de forma a calcular o tempo de propagação dos eventos de descoberta, que será de agora em diante referido como $TP_{descoberta}$ e calculado da seguinte forma. 

\begin{equation}
	\label{equ:performance}
	TP_{descoberta} = TS_{descoberta,app} - TS_{descoberta,\stwopa{}}
\end{equation}

Onde $TS_{descoberta,app}$ e $TS_{descoberta,\stwopa{}}$ se referem ao \timestamp{} de uma notificação de descoberta na aplicação e ao \timestamp{} do evento de descoberta gerado pelo \stwopa{}, respectivamente.

Para avaliar a acurácia é realizada uma comparação entre a quantidade de eventos que foram gerados e quantas notificações foram entregues à aplicação, e será calculada da seguinte maneira.

\begin{equation}
	\label{equ:acuracy}
	Acur\acute{a}cia = \frac{EventosGerados - |EventosGerados - EventosNotificados|}{EventosGerados}
\end{equation}

Onde $EventosGerados$ e $EventosNotificados$ se referem à quantidade de eventos gerados no \stwopa{} e à quantidade de notificações desses eventos que chegaram à aplicação, respectivamente.

A \autoref{tab:expected-results-1} mostra os valores do termo $EventosGerados$ na \autoref{equ:acuracy}, estes valores serão posteriormente comparados com o quantidade de eventos notificados para calcular a acurácia.

\begin{table}[htb]
	\begin{center}
		\IBGEtab{
			\caption{Resultados esperados do experimento 1}
			\label{tab:expected-results-1}
		}{
			\begin{tabular}{lc}
				\toprule
				& \textbf{$EventosGerados$}	    \\
				\midrule \midrule
				\textbf{Eventos de descoberta}	& 34814	    \\
				\textbf{Eventos de conexão}	& 0	    \\
				\textbf{Eventos de desconexão}	& 0	    \\
				\bottomrule
			\end{tabular}
		}{
			\fonte{\autoriapropria}
		}
	\end{center}
\end{table}

\subsection{Simulação dos \beacons{}}

O experimento descrito na \autoref{chap:avaliacao-experimento1} foi simulado utilizando um \dataset{}, esta seção descreve como este \dataset{} foi utilizado para simular a detecção dos \beacons{}. Os dados utilizados para a simulação do experimento 1 foram obtidos do \dataset{} disponibilizado por \citeonline{byrne2018residential}\footnote{O \dataset{} pode ser obtido em \cite{byrne2019dataset}}. Neste trabalho os autores instruíram que participantes conduzissem suas rotinas diárias em casa, enquanto sua localização na residência era monitorada.

O chão dos cômodos das casas foi marcado com etiquetas, cada etiqueta é uma imagem binária que codifica um número inteiro, e este, único para cada uma das etiquetas. Os participantes são equipados com uma câmera na região do torso que aponta em direção ao chão, detectando e interpretando qual etiqueta está visível no momento, e deste modo, identificando em qual cômodo o participante se encontra.

No trabalho citado, o autor realizou o experimento em 4 residências distintas, e estas, foram denotadas de ``\texttt{Residence A}'', ``\texttt{Residence B}'', ``\texttt{Residence C}'' e ``\texttt{Residence D}''. Para cada residência, o experimento foi conduzido múltiplas vezes, e cada um denominado de ``\texttt{living\_1}'', ``\texttt{living\_2}'', ``\texttt{living\_3}'', \dots{}, ``\texttt{living\_n}''.


A fim de maximizar a quantidade de eventos de descoberta, foi utilizado os dados do experimento ``\texttt{living\_2}'' da casa ``\texttt{Residence D}'' pois este experimento possui um dos maiores tempos de monitoramento e quantidade de entrada em cômodos. Os dados deste experimento consistem de 58 minutos de monitoramento de um indivíduo em uma casa de 10 cômodos. Os dados possuem uma resolução média de aproximadamente 11 medições por segundo.

Informações referentes à distribuição da quantidade de vezes em que o indivíduo visita um quarto estão sumarizadas na \autoref{fig:dataset-histogram}.

\begin{figure}[htb]
	
	\begin{center}
		
		\caption{\label{fig:dataset-histogram}Frequência de entrada por cômodo}
		\includegraphics[scale=0.246]{img/dataset-histogram}
		\fonte{\autoriapropria{}}
		
	\end{center}

\end{figure}

O \dataset{} original é composto principalmente por um arquivo de texto que contém em cada linha uma representação de um momento no decorrer do experimento. Entre os dados contidos por linha, destaca-se o \timestamp{} da medição e a etiqueta detectada naquele instante. 

Realizou-se um pré-processamento no \dataset{} de forma a facilitar a utilização dos dados para fim da simulação, gerando um arquivo contendo 3 valores separadas por espaço em que cada linha representa o instante em que o participante entrou e permaneceu contínuamente em um cômodo. Os 3 valores são: um identificador numérico do cômodo, o nome do cômodo e o \timestamp{} em que o indivíduo entrou naquele cômodo, como pode ser visto no \autoref{lst:preprocessed-dataset}.

\begin{center}
	\begin{lstlisting}[caption={Parte do \dataset{} pré-processado}, label=lst:preprocessed-dataset]
		#room	room-name	timestamp(seconds)
		2	living_area_A	485.352
		1	hallway_lower	506.273
		6	stairs		513.380
		8	hallway_upper	519.319
		10	bedroom_1	526.159
		8	hallway_upper	561.328
		7	bathroom_toilet	565.832
		8	hallway_upper	628.495
		9	bedroom_2	632.065
	\end{lstlisting}
\end{center}

Como descrito anteriormente, cada um dos 10 cômodos possui um \beacon{} \ble{}, configurados para emitir 10 anúncios por segundo. É esperado, então,  que o aplicativo detecte um anúncio a cada 0.1 segundo em que o participante permanece em um cômodo. 

Utilizando o \dataset{} já descrito, a cada momento em que o participante entra em determinado cômodo, calcula-se quantos anúncios o \beacon{} daquele cômodo transmitirá durante o tempo de permanência no local, o calculo é apresentado a seguir.

\begin{equation}
	anuncios = \frac{tempoPermanencia}{0.1} 
\end{equation}

Onde $tempoPermanencia$ denota o tempo que o indivíduo permaneceu no cômodo. Os anúncios são então repoduzidos a cada intervalo de 0.1 segundo. Com isso foi possível calcular previamente os valores da \autoref{tab:expected-results-1}.

\section{Coleta das métricas de performance}

Na \autoref{img:performance-annotation} é possível observar onde os \timestamps{} utilizados na avaliação de performance são anotados. Ambos os \timestamps{} são denotados na imagem por um cronômetro. Na imagem, $T0$ e $T1$ são equivalentes à $TS_{descoberta,\stwopa{}}$ e $TS_{descoberta,app}$ da \autoref{equ:performance}.

\begin{figure}
	\begin{center}
		\caption{\label{img:performance-annotation}Componentes onde há a captura de \timestamps{}}
		\includegraphics[scale = 0.50]{img/performance-annotation}
		\fonte{\autoriapropria{}}
	\end{center}
\end{figure}


\section{Experimento 2}

Para o cenário de uso 2 utilizou-se o cômodo ``\texttt{living\_area\_A}'' como a sala de estar onde o sistema multimídia se localiza. As informações referentes a quantidade de vezes que o participante entra em cada cômodo estão sumarizadas na \autoref{fig:dataset-histogram}.

Consiste em uma casa que possui um sistema multimídia na sala de estar, onde o conteúdo da televisão se adapta de acordo com os usuários que estão presentes na sala de estar, e a interação com as aplicações da televisão digital acontece através do \smartphone{}.

Ao entrar na sala de estar o sistema multimídia é encontrado e em seguida a conexão é estabelecida. Quando o usuário sai da sala de estar a conexão é desfeita. É medido o tempo de propagação de todos os eventos de descoberta, conexão e desconexão gerados no \stwopa{} para a aplicação.

\section{Descrição dos dados}

\section{Métricas}


Para o cenário de uso 2, a todo momento em que o participante entra na sala de estar acontece interações entre o aplicativo e o sistema multimídia, essas interações são compostas de um evento de descoberta, seguido por um evento de conexão. Ao sair do quarto um evento de desconexão é também gerado. 

As avaliaçãos de desempenho e acurácia deste cenário de uso são análogas às do cenário de uso 1, com a exceção de que eventos de conexão e desconexão também fazem partes das medições.

A \autoref{tab:expected-results-1} e a \autoref{tab:expected-results-2} mostram os resultados esperados de ambos os cenários de uso. Estes valores serão utilizados para a análise de acurácia da solução.

\begin{table}[htb]
	\begin{center}
		\IBGEtab{
			\caption{Resultados esperados do cenário de uso 2}
			\label{tab:expected-results-2}
		}{
			\begin{tabular}{lc}
				\toprule
				& \textbf{Cenário de uso 2 (Esperado)}	\\
				\midrule \midrule
				\textbf{Eventos de descoberta}	& 140	\\
				\textbf{Eventos de conexão}	& 140	\\
				\textbf{Eventos de desconexão}	& 139	\\
				\bottomrule
			\end{tabular}
		}{
			\fonte{\autoriapropria}
		}
	\end{center}
\end{table}

\section{Descrição dos experimentos}

Foi desenvolvido um aplicativo android utilizando o \mhubcddl{} onde o \smartphone{} permanece ativamente procurando e, quando possível, se conectando a \smartobjs{} no ambiente. O aplicativo é responsável por registrar a ocorrencia de interações com \smartobjs{}, juntamente com os \timestamps{} de cada um desses eventos. Os experimentos foram realizados no âmbito de dois cenários de uso descritos a seguir.

Os cenários de uso foram simulados com a tecnologia \fakesensor{} para gerar os eventos no \stwopa{}.



