\chapter{Introdução}

\section{Contextualização}

\subsection{\iot}

A Internet das Coisas (\textit{Internet of Things} -- \iot) é um paradigma de comunicação recente, onde objetos do dia a dia são equipados com equipamentos e protocolos que permitem que se comuniquem com seus usuários e com outros objetos, se tornando parte integral da Internet \cite{Atzori:2010}.  Pode ser definida como a interconexão de sensores e atuadores que fornece a capacidade de compartilhar informações entre plataformas através de um \framework{} unificado \cite{gubbi2013internet}. 

No contexto da \iot{} as ``coisas'' (do inglês: \textit{things}) são denominados ``\smartobjs'' \cite{bandyopadhyay2011internet}. Seu conceito se propõe a tornar a Internet ainda mais pervasiva e imersiva, ela promoverá o desenvolvimento de aplicações que utilizarão a grande quantidade e variedade de dados produzidos por esses objetos, de forma a prover serviços para os usuários. 

%colocar isso na Justificativa?
%A medida em que a \iot{} se expande em diversos setores da economia, a demanda para empregar e integrar dispositivos com capacidade de atuação também aumenta. Enquanto que muitos trabalhos lidem com plataformas de \middleware{} que deem suporte ao desenvolvimento de aplicações que necessitem de sensoriamento, pouco foi feito para prover serviços de atuação a nível de \middleware{} \cite{OJIOT_2018v4i1n03_Valim}.

No âmbito desse trabalho, é necessário definir o conceito de \middleware{}, componente essencial para aplicações de \iot{}.
\begin{citacao}
	O \middleware{} é uma camada de software ou conjunto de subcamadas interposta entre os níveis tecnológicos e de aplicação. Sua característica de esconder os detalhes de tecnologias diferentes é fundamental para isentar o programador de problemas que não são diretamente pertinentes para seu foco, que é o desenvolvimento da aplicação específica hablilitada pelas infraestruturas de \iot{} \cite[tradução~nossa]{Atzori:2010}.
\end{citacao}

\subsubsection{\iomt{}}

A \iot{} geralmente trata de \smartobjs{} estáticos, frequentemente presentes na infraestrutura do ambiente, como: sensores de uma sala, leitores \rfid{} de prédios inteligentes, etc. A Internet das Coisas Móveis (Internet of Mobile Things -- \iomt{}) é uma extensão da \iot{} clássica, onde os objetos e os \gateways{} são livres para se locomover, gerando uma maior dinamicidade de interações.

Exemplos de \smartobjs{} móveis incluem dispositivos vestíveis, veículos, e robôs móveis. Visto o contexto de mobilidade, \smartphones{} são adequados para assumir o papel de fornecedor de Internet e serviço de localização para \smartobjs{} que tenham disponíveis apenas tecnologias de comunicação sem fio de curto alcance e, dessa forma, não implementam a pilha TCP/IP \cite{talavera2015mobile}.

De acordo com \citeonline{nahrstedt2016internet} a \iomt{} se diferencia da \iot{} nos seguintes aspectos:
\begin{alineas}

	\item \emph{Contexto}, exemplo:
		
		\begin{alineas}

			\item Onde e com quem o dispositivo móvel se encontra.

		\end{alineas}

	\item \emph{Acesso à Internet e conectividade}, exemplo:
		
		\begin{alineas}

			\item Estado de conexão (conectado/desconectado);
				
			\item Se conectado, em qual rede.

		\end{alineas}

	\item \emph{Disponibilidade de energia}, exemplo:
		
		\begin{alineas}

			\item Onde o dispositivo pode ser recarregado;
				
			\item Quanta energia o aplicativo necessita.

		\end{alineas}

	\item \emph{Segurança e privacidade}, exemplo:
		
		\begin{alineas}

			\item Que tipo de infraestrutura de segurança o dispositivo encontra ao mudar de localização.

		\end{alineas}

\end{alineas}

\section{Caracterização do problema}


\section{Objetivos}


\section{Organização do texto}

O restante deste trabalho é estruturado da seguinte maneira: No \autoref{chap:fundamentacao} será apresentada a fundamentação teórica das tecnologias habilitadoras do trabalho. No \underline{Capítulo 3} a proposta do trabalho é definida. A avaliação da solução proposta é assunto do \underline{Capítulo 4}. As conclusões estão presentes no \underline{Capítulo 4}.

