\chapter{Introdução}

Após a leitura deste capítulo introdutório, o leitor será capaz de entender em qual contexto este trabalho está inserido, assim como compreender o problema e de que forma o trabalho se propõe a combatê-lo.

\section{Contextualização}

\subsection{\iot}

A Internet das Coisas (\textit{Internet of Things} -- \iot) é um paradigma de comunicação recente, onde objetos do dia a dia são equipados com equipamentos e protocolos que permitem que se comuniquem com seus usuários e com outros objetos, se tornando parte integral da Internet \cite{atzori:iera:morabito:2010}.  Pode ser definida como a interconexão de sensores e atuadores que fornece a capacidade de compartilhar informações entre plataformas através de um \framework unificado \cite{gubbi:et-al:2013}.

No contexto da \iot, as ``coisas'' (do inglês: \textit{things}) são denominados ``\smartobjs'' \cite{bandyopadhyay:sen:2011}. Seu conceito propõe tornar a Internet ainda mais pervasiva e imersiva, ela promoverá o desenvolvimento de aplicações que utilizarão a grande quantidade e variedade de dados produzidos por esses objetos, de forma a prover serviços para os usuários. 

%colocar isso na Justificativa?
%A medida em que a \iot{} se expande em diversos setores da economia, a demanda para empregar e integrar dispositivos com capacidade de atuação também aumenta. Enquanto que muitos trabalhos lidem com plataformas de \middleware{} que deem suporte ao desenvolvimento de aplicações que necessitem de sensoriamento, pouco foi feito para prover serviços de atuação a nível de \middleware{} \cite{OJIOT_2018v4i1n03_Valim}.

No âmbito desse trabalho, é necessário definir o conceito de \middleware, componente essencial para aplicações de \iot.
\begin{citacao}
	O \middleware é uma camada de software ou conjunto de subcamadas interposta entre os níveis tecnológicos e de aplicação. Sua característica de esconder os detalhes de diferentes tecnologias é fundamental para isentar o programador de problemas que não são diretamente pertinentes para seu foco, que é o desenvolvimento da aplicação específica habilitada pelas infraestruturas de \iot \cite[tradução~nossa]{atzori:iera:morabito:2010}.
\end{citacao}

\subsubsection{\iomt}

A \iot geralmente lida com \smartobjs estáticos, frequentemente presentes na infraestrutura do ambiente, como: sensores de uma sala, leitores \rfid de prédios inteligentes, etc. A Internet das Coisas Móveis (Internet of Mobile Things -- \iomt) é uma extensão da \iot clássica, onde os objetos e os \gateways são livres para se locomover, gerando uma maior dinamicidade de interações.

Exemplos de \smartobjs móveis incluem: dispositivos vestíveis, veículos, e robôs móveis. Visto o contexto de mobilidade, \smartphones são dispositivos adequados para assumir o papel de fornecedor de Internet e serviço de localização---ou seja, um \gateway---para \smartobjs que tenham disponíveis apenas tecnologias de comunicação sem fio de curto alcance e, dessa forma, não implementam a pilha TCP/IP \cite{talavera:et-al:2015}.

De acordo com \citeonline{nahrstedt:et-al:2016}, a \iomt se diferencia da \iot nos seguintes aspectos:
\begin{alineas}

	\item \emph{contexto}, exemplo:
		
	\begin{alineas}

		\item onde e com quem o dispositivo móvel se encontra.

	\end{alineas}

	\item \emph{acesso à Internet e conectividade}, exemplo:
		
	\begin{alineas}

		\item estado de conexão (conectado/desconectado);
			
		\item se conectado, em qual rede.

	\end{alineas}

	\item \emph{disponibilidade de energia}, exemplo:
		
	\begin{alineas}

		\item onde o dispositivo pode ser recarregado;
			
		\item quanta energia o aplicativo necessita.

	\end{alineas}

	\item \emph{segurança e privacidade}, exemplo:
		
	\begin{alineas}

		\item que tipo de infraestrutura de segurança o dispositivo encontra ao mudar de localização.

	\end{alineas}

\end{alineas}


\subsection{\mhubcddl}

O Mobile Hub/Context Data Distribution Layer (\mhubcddl) é um \middleware \iot para aquisição, processamento e distribuição de dados de contexto, com amplo suporte para criação de aplicações de \iot cientes de contexto que possuam requisitos de qualidade de contexto \cite{gomes:et-al:2017}.

Surgido de uma parceria entre o Laboratório de Sistemas Distribuídos Inteligentes (LSDi) da UFMA e o Laboratory for Advanced Collaboration (LAC) da PUC-Rio, o \middleware combina o \mhub \cite{talavera:et-al:2015} como \gateway móvel para aquisição dos dados de sensores, internos e externos, com o \cddl, um camada de distribuição responsável por registrar e descobrir serviços de contexto disponíveis, prover e monitorar dados de contexto e garantir a qualidade dos serviços de distribuição dos dados de contexto.

Adota o MQTT como único protocolo de comunicação, possuindo inclusive um microbroker interno. Assim, a captura dos dados de sensores obedece a arquitetura \textit{publish}/\textit{subscribe}.

\section{Caracterização do problema}

Dado a dinamicidade característica da \iomt, onde a topologia da rede muda constantemente e os \smartobjs e \gateways interagem de forma oportunística, torna-se claro a necessidade da existência de um mecanismo de notificação que informe à aplicação a ocorrência de eventos de descoberta, conexão e desconexão de \smartobjs com os \smartphones.

Um problema semelhante, entretanto mais desafiador, é a construção de tal mecanismo de forma que a aplicação seja notificada sobre os eventos que ocorrem no âmbito de outras aplicações que estão sendo executadas em outros dispositivos.

\section{Objetivos}

O trabalho terá como objetivo a proposta de um modelo do mecanismo de propagação de eventos de descoberta, conexão e desconexão de \smartobjs no \middleware \mhubcddl. A solução deverá se adequar aos requisitos de ambientes de \iomt.

\subsection{Objetivos específicos}

\begin{alineas}

	\item projetar a arquitetura do mecanismo de notificação;

	\item implementar a solução no \middleware \mhubcddl;

	\item realizar análises de performance e acurácia na solução proposta.

\end{alineas}


\section{Organização do texto}

O restante deste trabalho é estruturado da seguinte maneira: No \autoref{chap:fundamentacao} será apresentada a fundamentação teórica das tecnologias habilitadoras do trabalho. No \autoref{chap:solucao} a proposta do trabalho é definida. A avaliação da solução proposta é assunto do \autoref{chap:avaliacao}. As conclusões estão presentes no \autoref{chap:conclusao}.

