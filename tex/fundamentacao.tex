\chapter{Fundamentação teórica}
%\label{chap:fundamentacao}

\section{\iot}

A Internet das Coisas (\textit{Internet of Things} -- \iot) é um paradigma de comunicação recente, onde objetos do dia a dia são equipados com equipamentos e protocolos que permitem que se comuniquem com seus usuários e com outros objetos, se tornando parte integral da Internet \cite{Atzori:2010}.  Pode ser definida como a interconexão de sensores e atuadores que fornece a capacidade de compartilhar informações entre plataformas através de um \framework{} unificado \cite{gubbi2013internet}. 

No contexto da \iot{} as ``coisas'' (do inglês: \textit{things}) são denominados ``\smartobjs'' \cite{bandyopadhyay2011internet}. Seu conceito se propõe a tornar a Internet ainda mais pervasiva e imersiva, ela promoverá o desenvolvimento de aplicações que utilizarão a grande quantidade e variedade de dados produzidos por esses objetos, de forma a prover serviços para os usuários. 

%colocar isso na Justificativa?
A medida em que a \iot{} se expande em diversos setores da economia, a demanda para empregar e integrar dispositivos com capacidade de atuação também aumenta. Enquanto que muitos trabalhos lidem com plataformas de \middleware{} que deem suporte ao desenvolvimento de aplicações que necessitem de sensoriamento, pouco foi feito para prover serviços de atuação a nível de \middleware{} \cite{OJIOT_2018v4i1n03_Valim}.


