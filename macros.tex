% Define um novo termo com o comando \newterm
% \newterm{<nome do comando>}{<termo>}
% Este comando cria um novo comando de nome <nome do comando>, que representa o termo <termo>.
% O comando pode ser usado dentro do corpo do documento e será substituído pelo termo definido.
%
% Exemplo de uso:
% \newterm{\iot}{IoT}
% Agora, dentro do documento, ao usar \iot, ele será substituído por IoT.
% Você também pode definir um termo que inclua formatações extra. Exemplo:
% \newterm{\importantnote}{\textbf{Nota importante}} % negrito
% \newterm{\println}{\texttt{printf}} % monoespaçado
\newcommand*{\newterm}[2]{\newcommand*{#1}{#2\xspace}}

% Define um novo termo estrangeiro com o comando \newforeignterm
% \newforeignterm{<nome do comando>}{<termo em itálico>}
% Funciona como o comando \newterm, mas automaticamente formata o termo em itálico.
% Exemplo de uso:
% \newforeignterm{\dataset}{dataset}
% Agora, dentro do documento, ao usar \dataset, ele será substituído por dataset em itálico.
\newcommand*{\newforeignterm}[2]{\newterm{#1}{\textit{#2}}}

% Defina aqui os termos comumente usados no seu documento onde você quer garantir uma escrita e/ou formatação consistente; Sinta-se livre para organizar como bem entender.
% Algumas palavras estrangeiras são consideradas comuns o suficiente de forma que é aceito e aconselhável que não sejam grafadas com itálico, não existe uma regra, use o bom senso. Você também pode seguir o manual de comunicação da SECOM como referência: https://www12.senado.leg.br/manualdecomunicacao/estilos/estrangeirismos-grafados-sem-italico-ou-aspas

% Conceitos
\newterm{\api}{API}
\newterm{\framework}{framework}
\newterm{\gateway}{gateway}
\newterm{\Gateway}{Gateway}
\newterm{\gateways}{gateways}
\newterm{\Gateways}{Gateways}
\newterm{\iomt}{IoMT}
\newterm{\iot}{IoT}
\newterm{\middleware}{middleware}
\newterm{\Middleware}{Middleware}
\newterm{\pubsub}{\pub/\sub}
\newterm{\rendezvous}{rendez-vous}
\newterm{\smartphone}{smart\-phone}
\newterm{\Smartphone}{Smart\-phone}
\newterm{\smartphones}{smart\-phones}
\newterm{\Smartphones}{Smart\-phones}
\newterm{\software}{software}
\newterm{\Software}{Software}

\newforeignterm{\broadcast}{broadcast}
\newforeignterm{\broker}{broker}
\newforeignterm{\Broker}{Broker}
\newforeignterm{\brokers}{brokers}
\newforeignterm{\Brokers}{Brokers}
\newforeignterm{\dataset}{dataset}
\newforeignterm{\Dataset}{Dataset}
\newforeignterm{\listener}{listener}
\newforeignterm{\pub}{publisher}
\newforeignterm{\pubs}{publishers}
\newforeignterm{\smartobj}{smart object}
\newforeignterm{\Smartobj}{Smart object}
\newforeignterm{\smartobjs}{smart objects}
\newforeignterm{\Smartobjs}{Smart objects}
\newforeignterm{\subs}{subscribers}
\newforeignterm{\sub}{subscriber}
\newforeignterm{\timestamps}{timestamps}
\newforeignterm{\timestamp}{timestamp}
\newforeignterm{\ubroker}{microbroker}

% Tecnologias
\newterm{\ble}{BLE}
\newterm{\mqtt}{MQTT}
\newterm{\rfid}{RFID}

\newforeignterm{\beacon}{beacon}
\newforeignterm{\Beacon}{Beacon}
\newforeignterm{\beacons}{beacons}
\newforeignterm{\Beacons}{Beacons}
\newforeignterm{\bluetooth}{bluetooth}
\newforeignterm{\Bluetooth}{Bluetooth}
\newforeignterm{\BluetoothLowEnergy}{Bluetooth Low Energy}

% Substantivos próprios
\newterm{\android}{Android}
\newterm{\cddl}{CDDL}
\newterm{\eventbus}{EventBus}
\newterm{\faketechnology}{\texttt{FakeTechnology}}
\newterm{\mhubcddl}{M-Hub/CDDL}
\newterm{\mhub}{M-Hub}
\newterm{\msg}{\texttt{Message}}
\newterm{\objconnectedmsg}{\texttt{ObjectConnectedMessage}}
\newterm{\objdisconnectedmsg}{\texttt{ObjectDisconnectedMessage}}
\newterm{\objfoundmsg}{\texttt{ObjectFoundMessage}}
\newterm{\qocevaluator}{\texttt{QoCEvaluator}}
\newterm{\sensordatamsg}{\texttt{SensorDataMessage}}
\newterm{\sensordata}{\texttt{SensorData}}
\newterm{\stwopa}{S2PA}
\newterm{\stwopaservice}{\texttt{S2PAService}}
\newterm{\techinterface}{\texttt{Technology}}
\newterm{\techlistener}{\texttt{TechnologyListener}}

% Utils
\newterm{\autoriapropria}{Produzido pelo autor}

\titulo{Descoberta e Desconexão de Objetos Inteligentes (\textit{Smart Objects}) em Ambientes Oportunísticos de IoMT}
\autor{Alysson Cirilo Silva}
\data{2019}
\instituicao{Universidade Federal do Maranhão}
\local{São Luís -- MA}
\preambulo{Monografia apresentada ao curso de Ciência da Computação da Universidade Federal do Maranhão, como parte dos requisitos necessários para obtenção do grau de Bacharel em Ciência da Computação.}
\tipotrabalho{Monografia (Graduação)}
\orientador{Francisco José da Silva e Silva}

